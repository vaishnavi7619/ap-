\let\negmedspace\undefined
\let\negthickspace\undefined
\documentclass[journal,12pt,twocolumn]{IEEEtran}

\usepackage{cite}
\usepackage{amsmath,amssymb,amsfonts,amsthm}
\usepackage{algorithmic}
\usepackage{graphicx}
\usepackage{textcomp}
\usepackage{xcolor}
\usepackage{txfonts}
\usepackage{listings}
\usepackage{enumitem}
\usepackage{mathtools}
\usepackage{gensymb}
\usepackage[breaklinks=true]{hyperref}
\usepackage{tkz-euclide} % loads  TikZ and tkz-base
\usepackage{listings}
\usepackage{circuitikz}
\usepackage{graphicx}

%\newcounter{MYtempeqncnt}
\DeclareMathOperator*{\Res}{Res}
%\renewcommand{\baselinestretch}{2}
\renewcommand\thesection{\arabic{section}}
\renewcommand\thesubsection{\thesection.\arabic{subsection}}
\renewcommand\thesubsubsection{\thesubsection.\arabic{subsubsection}}

\renewcommand\thesectiondis{\arabic{section}}
\renewcommand\thesubsectiondis{\thesectiondis.\arabic{subsection}}
\renewcommand\thesubsubsectiondis{\thesubsectiondis.\arabic{subsubsection}}

% correct bad hyphenation here
\hyphenation{op-tical net-works semi-conduc-tor}
\def\inputGnumericTable{}                                 %%

\lstset{
	frame=single,
	breaklines=true,
	columns=fullflexible
}

\newtheorem{theorem}{Theorem}[section]
\newtheorem{problem}{Problem}
\newtheorem{proposition}{Proposition}[section]
\newtheorem{lemma}{Lemma}[section]
\newtheorem{corollary}[theorem]{Corollary}
\newtheorem{example}{Example}[section]
\newtheorem{definition}[problem]{Definition}
\newcommand{\BEQA}{\begin{eqnarray}}
	\newcommand{\EEQA}{\end{eqnarray}}
\newcommand{\define}{\stackrel{\triangle}{=}}
\newcommand\figref{Fig.~\ref}
\newcommand\tabref{Table~\ref}
\bibliographystyle{IEEEtran}
%\bibliographystyle{ieeetr}


\providecommand{\mbf}{\mathbf}
\providecommand{\pr}[1]{\ensuremath{\Pr\left(#1\right)}}
\providecommand{\qfunc}[1]{\ensuremath{Q\left(#1\right)}}
\providecommand{\sbrak}[1]{\ensuremath{{}\left[#1\right]}}
\providecommand{\lsbrak}[1]{\ensuremath{{}\left[#1\right.}}
\providecommand{\rsbrak}[1]{\ensuremath{{}\left.#1\right]}}
\providecommand{\brak}[1]{\ensuremath{\left(#1\right)}}
\providecommand{\lbrak}[1]{\ensuremath{\left(#1\right.}}
\providecommand{\rbrak}[1]{\ensuremath{\left.#1\right)}}
\providecommand{\cbrak}[1]{\ensuremath{\left\{#1\right\}}}
\providecommand{\lcbrak}[1]{\ensuremath{\left\{#1\right.}}
\providecommand{\rcbrak}[1]{\ensuremath{\left.#1\right\}}}
\theoremstyle{remark}
\newtheorem{rem}{Remark}
\newcommand{\sgn}{\mathop{\mathrm{sgn}}}
\providecommand{\abs}[1]{\left\vert#1\right\vert}
\providecommand{\res}[1]{\Res\displaylimits_{#1}}
\providecommand{\norm}[1]{\left\lVert#1\right\rVert}
%\providecommand{\norm}[1]{\lVert#1\rVert}
\providecommand{\mtx}[1]{\mathbf{#1}}
\providecommand{\mean}[1]{E\left[ #1 \right]}
\providecommand{\fourier}{\overset{\mathcal{F}}{ \rightleftharpoons}}
%\providecommand{\hilbert}{\overset{\mathcal{H}}{ \rightleftharpoons}}
\providecommand{\system}{\overset{\mathcal{H}}{ \longleftrightarrow}}
%\newcommand{\solution}[2]{\textbf{Solution:}{#1}}
\newcommand{\solution}{\noindent \textbf{Solution: }}
\newcommand{\cosec}{\,\text{cosec}\,}
\providecommand{\dec}[2]{\ensuremath{\overset{#1}{\underset{#2}{\gtrless}}}}
\newcommand{\myvec}[1]{\ensuremath{\begin{pmatrix}#1\end{pmatrix}}}
\newcommand{\mydet}[1]{\ensuremath{\begin{vmatrix}#1\end{vmatrix}}}
\renewcommand{\abstractname}{Question}

\let\vec\mathbf

\vspace{3cm}


\newcommand{\permcomb}[4][0mu]{{{}^{#3}\mkern#1#2_{#4}}}
\newcommand{\comb}[1][-1mu]{\permcomb[#1]{C}}

%\IEEEpeerreviewmaketitle

\newcommand \tab [1][1cm]{\hspace*{#1}}
%\newcommand{\Var}{$\sigma ^2$}

\title{
	
	\title{NCERT Maths 11.9.2 Q9}
	\author{EE23BTECH11014- DEVARAKONDA GUNA VAISHNAVI$^{*}$% <-this % stops a space
	}
	
	
}
\begin{document}
\maketitle
	
\textbf{Question:} 
The sum of the first $n$ terms of two arithmetic progressions (AP) is in the ratio $5n+4 : 9n+6$. Find the ratio of their 18th terms.

solution:



\begin{table}[h!]
    \centering
    \input{tables/ ap tables.tex}
    \caption{Input Parameters}
    \label{table:parameters}
\end{table}



\begin{align}
x_1(n)=(x_1(0)+nd_1)u(n)
\label{eq:1}
\end{align}
\begin{align}
x_2(n)=(x_2(0)+nd_2)u(n)
\label{eq:2}
\end{align}
Applying Z transform:
\begin{align}
    x_1(z) &=\frac{x_1(0)}{1-z^{-1}} + \frac{d_1z^{-1}}{(1-z^{-1})^2}\\
     x_2(z)&= \frac{x_2(0)}{1-z^{-1}} + \frac{d_2z^{-1}}{(1-z^{-1})^2}
\end{align}
Region of Convergence or R.O.C :
\begin{align}
     \abs{z}>1
\end{align}
For AP, the sum of first n+1 terms can be written as :
\begin{align}
	 y(n)&=x(n)*u(n)
\end{align}  
Applying Z transform on both sides
\begin{align}
 x_1(z) &=\frac{x_1(0)}{(1-z^{-1})^2} + \frac{d_1z^{-1}}{(1-z^{-1})^3}\\
x_2(z)&= \frac{x_2(0)}{(1-z^{-1})^2} + \frac{d_2z^{-1}}{(1-z^{-1})^3}
\end{align}
 

Using contour integration to find inverse Z transform:
\begin{align}
	y_1(n) &= \frac{1}{2\pi j} \oint_C Y(z) z^{n-1} dz\\
	&= \frac{1}{2\pi j} \oint_C \left[ \frac{x_1(0)}{(1-z^{-1})^2} - \frac{d_1z^{-1}}{(1-z^{-1})^3} \right]z^{n-1} \, dz
 \end{align}
 \begin{align}
 y_2(n) &= \frac{1}{2\pi j} \oint_C Y(z) z^{n-1} dz\\
	&= \frac{1}{2\pi j} \oint_C \left[ \frac{x_2(0)}{(1-z^{-1})^2} - \frac{d_2z^{-1}}{(1-z^{-1})^3} \right]z^{n-1} \, dz
 \end{align}
 The sum of the terms of the sequence is computed using the residue theorem, expressed as $R_i$, which represents the residue of the Z-transform at $ z=1 $ for the expression $ Y(z) $.
\begin{align}
	R_i=R_1 + R_2
\end{align}
 $R_1$ and $R_2$ are residues calculated at the poles of the Z-transform.
\begin{align}
for  y_1(n)
\vspace{1cm}
R_1 &= \frac{1}{{(2-1)!}} \left. \frac{d (x_1(0)z^{n+1})}{dz} \right|_{z=1} \
\end{align}
\begin{align}
R_2&=\frac{1}{{(3-1)!}} \left. \frac{d^2(d_1z^{n+1})}{dz^2} \right|_{z=1} 
\end{align}
The sum of terms is given by $R_i$:
 \begin{align}
y_1(n)=x_1(0)(n+1)	+ \frac{d_1}{2} n(n+1) 
 \end{align}
\begin{align}
for y_2(n)
R_1& = \frac{1}{{(2-1)!}} \left. \frac{d (x_2(0)z^{n+1})}{dz} \right|_{z=1} \
\end{align}
\begin{align}
R_2&=\frac{1}{{(3-1)!}} \left. \frac{d^2(d_2z^{n+1})}{dz^2} \right|_{z=1} 
\end{align}

The sum of terms is given by $R_i$:
\begin{align}
 y_2(n)=x_2(0)(n+1)	+ \frac{d_2}{2} n(n+1) 
 \end{align}

taking ratio of $y_1(n)$:$y_2(n)$ and equating to $5(n+1)+4$:$9(n+1)+6$\\
\begin{align}
\frac{y_1(n)}{y_2(n)}&=\frac{x_1(0)(n+1)+ \frac{d_1}{2} n(n+1)}{x_2(0)(n+1)+ \frac{d_2}{2} n(n+1)}\\
&=\frac{x_1(0)+\frac{d_1}{2}{n}}{x_2(0)+\frac{d_2}{2}{n}}
\end{align}
we need ratio of 18th  terms $\frac{n}{2}$ is 17 and n=34 
substitute it in $5(n+1)+4$:$9(n+1)+6$ then we get ratio as $\frac{179}{321}
$


	

 

\end{document}

